\documentclass[a4paper, 12pt]{article}
\usepackage{fontspec}
\usepackage{xeCJK}
\usepackage{mathtools}
\usepackage{xcolor}

% \setmainfont{新細明體}
\setCJKmainfont{新細明體}

\XeTeXlinebreaklocale "zh"

\title{算法班手寫作業 10}
\author{李緒成}

\begin{document}
    \maketitle
    \newpage
    \begin{enumerate}
        \item 證明lowbit$(x) =x\&(−x)$
            \begin{itemize}
                \item 設$x>0$,設$x$的二進位表示法中,第$x$位為$1$,第 $0$ 到第 $k-1$ 位都為 $0$ 
                \item 對x的二進位表示法取反($\sim x$),可以得到$\sim x$的二進位表示法中,第$k$位為$0$,第$0$到第$k-1$位都為$1$
                \item 得到$\sim x+1$的二進位表示法的第 $k+1$ 位至其最高位都為與 $x$ 的二進位表示法中相反的數字
                \item 而$\sim x+1$的二進位表示法的第$k$為$1$,第$0$至第$k-1$位都為$0$
                \item 且$x$的二進位表示法的第$k$位為$1$
                \item 所以將 $\sim x+1$ 與$x$ 進行$\&$運算後,即可得到$x$的lowbit 
                \item 又$-x=\sim x+1$ ,所以 lowbit$(x)= x \& (−x)$
            \end{itemize}
        \newpage
        \item
            \begin{enumerate}
                \item 證明$\sum\limits_{i = 0}^{n} i·2^{i−1}= (n−1)·2^n+ 1$
                    \begin{enumerate}
                        \item 證$n = 1$時成立:\\
                              帶入後得到$$\sum_{i = 0}^{1} i·2^{i−1} = (1−1)·2^1+ 1 = 1$$
                              命題成立
                        \item 證若在$n=k$時成立, 則$n=k+1$時也會成立:\\
                              假設$n=k$時成立,即$$\sum_{i = 0}^{k} i·2^{i−1}= (k−1)·2^k+ 1$$
                              則當$n=k+1$時
                              \begin{align*}
                                &\sum_{i = 0}^{k+1} i·2^{i−1}= ((k+1)−1)·2^{k+1}+ 1\\
                                &\sum_{i = 0}^{k} i·2^{i−1} + (k+1) \cdot 2^k = 2k·2^{k}+ 1\\
                                &\sum_{i = 0}^{k} i·2^{i−1} + (k+1) \cdot 2^k = (k−1)·2^k + (k+1)·2^k + 1\\
                                &\sum_{i = 0}^{k} {i}·2^{i−1} + {\color{red}(k+1) \cdot 2^k} = (k−1)·2^k + 1 + {\color{red}(k+1) \cdot 2^k}
                              \end{align*}
                              
                              命題仍然成立\\
                              由i, ii以數學歸納法得證
                    \end{enumerate}
                \newpage
                \item 證明:$f(m)\geq m·2^{m−1},$ for $ m\geq 0。($即:$f(m)\geq n^2·log_{2} n, $ for $ n= 2^m, m\geq 0)$
                    \begin{enumerate}
                        \item 證$m = 1$時成立:\\
                              帶入後得到
                              \begin{align*}
                                &f(1)=2^1+\sum_{k=0}^{0} f(k)\geq 1 \cdot 2^0 \\\\
                                &3\geq 1 
                              \end{align*}
                              命題成立
                        \item 證若在$n=t$時成立, 則$n=t+1$時也會成立:\\
                              假設$m=t$時成立,即$$f(t)=2^{t}+\sum_{k=0}^{t-1} f(k)\geq t·2^{t−1}$$
                              則當$n=t+1$時
                              \begin{align*}
                                f(t+1) &= 2^{t+1}+\sum_{k=0}^{t} f(k)\geq (t+1)·2^{t}\\
                                       &= 2^{t+1}+\sum_{k=0}^{t} f(k)\geq (2t+2)·2^{t-1}\\
                                       &= 2^{t+1}+\sum_{k=0}^{t} f(k)\geq 2t·2^{t-1}+2·2^{t-1}\\
                                       &= 2^{t+1}+f(t)+\sum_{k=0}^{t-1} f(k)\geq 2t·2^{t-1}+2^{t}\\
                                       &= f(t)+2^{t}+{\color{red}2^{t}+\sum_{k=0}^{t-1} f(k)}\geq 2t·2^{t-1}+2^{t}\\
                                       &= f(t)+2^{t}+{\color{red}f(t)}\geq 2t·2^{t-1}+2^{t}\\\\
                                       &= 2\cdot f(t)+2^{t}\geq 2\cdot t·2^{t-1}+2^{t}
                              \end{align*}
                              命題仍然成立\\
                              由i, ii以數學歸納法得證
                \newpage

                    \end{enumerate}
            \end{enumerate}
        \item 
            \begin{enumerate}
                \item query$(2, 8)$或是假設有n個元素,則query$(2, n)$
                \item $O((\log_2 n)^2)$
            \end{enumerate}
        \item 
            \begin{enumerate}
                \item ans = query$($dif$, x)$
                \item modify$($dif$, b+1, -$val$), $modify$($dif2$, b+1, -$val$\cdot (b+1));$\\
                        modify(dif$, a, $val), modify(dif2$, a, $val$\cdot a)$;
                \item $\sum\limits_{i=1}^{x} arr[i] = \sum\limits_{i=1}^{x} (x-i+1)$dif$[i] $
                \item query$($dif$, x)\cdot (x+1) - $query$($dif2$, x)$
            \end{enumerate}
    \end{enumerate}

\end{document} 